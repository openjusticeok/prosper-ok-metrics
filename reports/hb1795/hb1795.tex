% Options for packages loaded elsewhere
\PassOptionsToPackage{unicode}{hyperref}
\PassOptionsToPackage{hyphens}{url}
\PassOptionsToPackage{dvipsnames,svgnames,x11names}{xcolor}
%
\documentclass[
  letterpaper,
  DIV=11,
  numbers=noendperiod]{scrartcl}

\usepackage{amsmath,amssymb}
\usepackage{iftex}
\ifPDFTeX
  \usepackage[T1]{fontenc}
  \usepackage[utf8]{inputenc}
  \usepackage{textcomp} % provide euro and other symbols
\else % if luatex or xetex
  \usepackage{unicode-math}
  \defaultfontfeatures{Scale=MatchLowercase}
  \defaultfontfeatures[\rmfamily]{Ligatures=TeX,Scale=1}
\fi
\usepackage{lmodern}
\ifPDFTeX\else  
    % xetex/luatex font selection
\fi
% Use upquote if available, for straight quotes in verbatim environments
\IfFileExists{upquote.sty}{\usepackage{upquote}}{}
\IfFileExists{microtype.sty}{% use microtype if available
  \usepackage[]{microtype}
  \UseMicrotypeSet[protrusion]{basicmath} % disable protrusion for tt fonts
}{}
\makeatletter
\@ifundefined{KOMAClassName}{% if non-KOMA class
  \IfFileExists{parskip.sty}{%
    \usepackage{parskip}
  }{% else
    \setlength{\parindent}{0pt}
    \setlength{\parskip}{6pt plus 2pt minus 1pt}}
}{% if KOMA class
  \KOMAoptions{parskip=half}}
\makeatother
\usepackage{xcolor}
\setlength{\emergencystretch}{3em} % prevent overfull lines
\setcounter{secnumdepth}{5}
% Make \paragraph and \subparagraph free-standing
\makeatletter
\ifx\paragraph\undefined\else
  \let\oldparagraph\paragraph
  \renewcommand{\paragraph}{
    \@ifstar
      \xxxParagraphStar
      \xxxParagraphNoStar
  }
  \newcommand{\xxxParagraphStar}[1]{\oldparagraph*{#1}\mbox{}}
  \newcommand{\xxxParagraphNoStar}[1]{\oldparagraph{#1}\mbox{}}
\fi
\ifx\subparagraph\undefined\else
  \let\oldsubparagraph\subparagraph
  \renewcommand{\subparagraph}{
    \@ifstar
      \xxxSubParagraphStar
      \xxxSubParagraphNoStar
  }
  \newcommand{\xxxSubParagraphStar}[1]{\oldsubparagraph*{#1}\mbox{}}
  \newcommand{\xxxSubParagraphNoStar}[1]{\oldsubparagraph{#1}\mbox{}}
\fi
\makeatother


\providecommand{\tightlist}{%
  \setlength{\itemsep}{0pt}\setlength{\parskip}{0pt}}\usepackage{longtable,booktabs,array}
\usepackage{calc} % for calculating minipage widths
% Correct order of tables after \paragraph or \subparagraph
\usepackage{etoolbox}
\makeatletter
\patchcmd\longtable{\par}{\if@noskipsec\mbox{}\fi\par}{}{}
\makeatother
% Allow footnotes in longtable head/foot
\IfFileExists{footnotehyper.sty}{\usepackage{footnotehyper}}{\usepackage{footnote}}
\makesavenoteenv{longtable}
\usepackage{graphicx}
\makeatletter
\def\maxwidth{\ifdim\Gin@nat@width>\linewidth\linewidth\else\Gin@nat@width\fi}
\def\maxheight{\ifdim\Gin@nat@height>\textheight\textheight\else\Gin@nat@height\fi}
\makeatother
% Scale images if necessary, so that they will not overflow the page
% margins by default, and it is still possible to overwrite the defaults
% using explicit options in \includegraphics[width, height, ...]{}
\setkeys{Gin}{width=\maxwidth,height=\maxheight,keepaspectratio}
% Set default figure placement to htbp
\makeatletter
\def\fps@figure{htbp}
\makeatother

\usepackage{booktabs}
\usepackage{longtable}
\usepackage{array}
\usepackage{multirow}
\usepackage{wrapfig}
\usepackage{float}
\usepackage{colortbl}
\usepackage{pdflscape}
\usepackage{tabu}
\usepackage{threeparttable}
\usepackage{threeparttablex}
\usepackage[normalem]{ulem}
\usepackage{makecell}
\usepackage{xcolor}
\usepackage{caption}
\usepackage{anyfontsize}
\KOMAoption{captions}{tableheading}
\makeatletter
\@ifpackageloaded{caption}{}{\usepackage{caption}}
\AtBeginDocument{%
\ifdefined\contentsname
  \renewcommand*\contentsname{Table of contents}
\else
  \newcommand\contentsname{Table of contents}
\fi
\ifdefined\listfigurename
  \renewcommand*\listfigurename{List of Figures}
\else
  \newcommand\listfigurename{List of Figures}
\fi
\ifdefined\listtablename
  \renewcommand*\listtablename{List of Tables}
\else
  \newcommand\listtablename{List of Tables}
\fi
\ifdefined\figurename
  \renewcommand*\figurename{Figure}
\else
  \newcommand\figurename{Figure}
\fi
\ifdefined\tablename
  \renewcommand*\tablename{Table}
\else
  \newcommand\tablename{Table}
\fi
}
\@ifpackageloaded{float}{}{\usepackage{float}}
\floatstyle{ruled}
\@ifundefined{c@chapter}{\newfloat{codelisting}{h}{lop}}{\newfloat{codelisting}{h}{lop}[chapter]}
\floatname{codelisting}{Listing}
\newcommand*\listoflistings{\listof{codelisting}{List of Listings}}
\makeatother
\makeatletter
\makeatother
\makeatletter
\@ifpackageloaded{caption}{}{\usepackage{caption}}
\@ifpackageloaded{subcaption}{}{\usepackage{subcaption}}
\makeatother

\ifLuaTeX
  \usepackage{selnolig}  % disable illegal ligatures
\fi
\usepackage{bookmark}

\IfFileExists{xurl.sty}{\usepackage{xurl}}{} % add URL line breaks if available
\urlstyle{same} % disable monospaced font for URLs
\hypersetup{
  pdftitle={HB 1795 Impact Estimate},
  pdfauthor={Polina Rozhkova},
  colorlinks=true,
  linkcolor={blue},
  filecolor={Maroon},
  citecolor={Blue},
  urlcolor={Blue},
  pdfcreator={LaTeX via pandoc}}


\title{HB 1795 Impact Estimate}
\usepackage{etoolbox}
\makeatletter
\providecommand{\subtitle}[1]{% add subtitle to \maketitle
  \apptocmd{\@title}{\par {\large #1 \par}}{}{}
}
\makeatother
\subtitle{Open Justice Oklahoma}
\author{Polina Rozhkova}
\date{2024-07-16}

\begin{document}
\maketitle
\begin{abstract}
Between 9,187 and 16,116 Oklahomans with misdemeanor drug charges were
impacted by provisions in HB 1795 between January 1, 2022 and December
1, 2023.
\end{abstract}


\section{Background}\label{background}

Oklahoma's
\href{http://webserver1.lsb.state.ok.us/cf_pdf/2021-22\%20ENR/hB/HB1795\%20ENR.PDF}{House
Bill 1795} went into effect November 1, 2021 (Q2 FY 2022). This new law
reduces the number of offenses that can lead to a license revocation,
although certain circumstances may still result in revocation.
Additionally, HB 1795 sought to enhance the accessibility and
affordability of provisional licenses. The legislation imposed
restrictions on license suspensions resulting from ``failure to pay''
and incorporated a significant provision reducing the minimum payment
plan for outstanding fines and fees. Previously set at \$25 per month,
HB 1795 lowers the minimum monthly payment to \$5. The aforementioned
fees encompass various charges such as provisional license fees, warrant
fees, court costs or fees, and driver license reinstatement fees.

\section{Data and Methodology}\label{data-and-methodology}

Open Justice Oklahoma maintains a database of administrative court
records which includes information on all criminal misdemeanors and
felonies filed in Oklahoma beginning in 2001. Case information is
systematically collected from publicly available data hosted on the
\href{https://www.oscn.net/v4/}{Oklahoma State Court Network (OSCN)}
website. This analysis uses Oklahoma court records beginning January 1,
2022 (Q3 FY 2022) to estimate the impact of HB 1795 during a period of
full implementation. We explore two provisions of the bill in order to
estimate lower and upper bounds for the number of individuals with
misdemeanor drug charges that were impacted by the bill's passing.

To estimate the lower bound, we focus on the provision concerned with
modifying the types of offenses requiring immediate revocation of
driving privileges (Section 6-205). Using the Department of Safety
\href{https://oklahoma.gov/content/dam/ok/en/dps/VCB\%20February\%202022.pdf}{violation
codes} we pull court records for individuals who are charged with
possession of a controlled dangerous substance as defined by Oklahoma's
\href{http://www.oklegislature.gov/cf_pdf/2003-04\%20INT/hb/HB2166\%20int.pdf}{Uniform
Controlled Dangerous Substances Act} while driving a motor vehicle.
Court records also include drug violation codes that are similar but do
not specify use of a motor vehicle. These violation codes provide a
minimum estimate of individuals whose drug charges would have resulted
in immediate driver license suspension or revocation prior to the
enactment of HB 1795.

Prior to the bill's passing, individuals could have their license
suspended or revoked for the failure to pay court fines or fees. Some of
these fines and fees are imposed on individuals who are charged with
misdemeanor drug offenses. Following this logic, we account for any
individuals who may be subject to fines and/or fees due to a misdemeanor
drug possession charge. We combine the previously calculated lower bound
and adjust for duplicate entries (individuals that may appear in both
methods) for an upper bound estimate.

\section{Results}\label{results}

\begingroup
\fontsize{12.0pt}{14.4pt}\selectfont
\begin{longtable*}{rr}
\caption*{
{\large Estimated Impact for Oklahomans with Misdemeanor Drug Charges} \\ 
{\small 2022-01-01 to 2024-07-01}
} \\ 
\toprule
Upper Bound & Lower Bound \\ 
\midrule\addlinespace[2.5pt]
{22859} & {10968} \\ 
\bottomrule
\end{longtable*}
\endgroup




\end{document}
